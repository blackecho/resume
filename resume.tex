% LaTeX file for resume
% This file uses the resume document class (res.cls)

\documentclass[margin]{res}
% the margin option causes section titles to appear to the left of body text
\textwidth=5.2in % increase textwidth to get smaller right margin
%\usepackage{helvetica} % uses helvetica postscript font (download helvetica.sty)
%\usepackage{newcent}   % uses new century schoolbook postscript font
\usepackage{fontawesome}
\usepackage{hyperref}

\begin{document}
    \name{Gabriele Angeletti\\[12pt]} % the \\[12pt] adds a blank line after name
    \address{London, UK  \\ (+44) 7803 056685}
    \begin{resume}
        \section{}
        \faEnvelope~\textbf{angeletti.gabriele@gmail.com} \\[5pt]
        \faGithub~\textbf{\url{https://github.com/blackecho}} \\[5pt]
        \faLinkedin~\textbf{\url{https://www.linkedin.com/in/gabriele-angeletti}} \\[5pt]
        \faSkype~\textbf{gabriele.angeletti1}

        \section{Experience}
            {\bf Software Engineer,} \href{https://www.lyft.com/self-driving-vehicles/}{Lyft Level 5}
            \hfill Oct 2018 -- Present\\
            Working on distributed pipelines to turn large-scale imagery data into accurate 3D visual maps.\\
            Automate maps quality assurance by leveraging tools such as BigQuery and Datastudio.\\
            Development of visualization tools that works at scale and help identifying driving patterns.

            {\bf Research Engineer,} \href{http://www.bluevisionlabs.com}{Blue Vision Labs} \hfill Feb 2018 -- Oct 2018\\
            Design of data pipelines to analyze the behaviour of 3D mapping and localization.\\
            Work on distributed algorithms to perform 3D mapping at scale.\\
            Research on using machine learning throughout our mapping system to improve quality.

            {\bf Research Engineer intern,} \href{http://www.bluevisionlabs.com}{Blue Vision Labs} \hfill Aug 2017 -- Dec 2017\\
            Research on replacing SIFT with deep learning to improve local feature description.\\
	        Design of tools to assess the performance of our 3D mapping system.\\
	        Work on a data warehouse solution that scales to billions of records.

            {\bf Contractor,} \hfill 2014 -- 2017\\
            Part-time work on different projects during university, mainly in full-stack web development, web crawling, and natural language processing.

            {\bf Student,} \href{http://www.innovactionlab.org/?lang=en}{InnovAction Lab} \hfill Spring 2013\\
            Entrepreneurship course about how to build startups and present business ideas directly to investors.

        \section{Education}
            Sapienza University of Rome, \hfill 2015 -- 2017 \\
            M.Sc.\ in Engineering in Artificial Intelligence and Robotics, (English Degree) \\
            Final grade 110/110 \\
            Thesis: \textit{Adaptive Deep Learning through Visual Domain Localization \footnote{\url{https://github.com/blackecho/master-thesis}}}

            Sapienza University of Rome, \hfill 2011 -- 2014 \\
            B.Sc.\ in Engineering in Computer Science and Automation, (Italian Degree) \\
            Final grade 106/110 \\
            Thesis: \textit{Statistical analysis of mobile apps reviews to improve users' QoE}

        \section{Software engineering}
        \begin{description}
            \item \textbf{Python}: SciPy ecosystem, PyTorch, TensorFlow, PySpark
            \item \textbf{Data engineering}: Spark, Airflow, Luigi
            \item \textbf{Databases}: SQL, AWS RedShift, Google BigQuery
            \item \textbf{Cloud technologies}: experience with AWS and GCP main services
            \item \textbf{Front-end}: ES6, TypeScript, Angular
            \item \textbf{Golang}: little experience with building microservices
            \item \textbf{DevOps}: familiar with Ansible, Terraform, and Docker
        \end{description}

        \section{Projects}
        \begin{description}
            \item Deep Learning TensorFlow \footnote{\url{https://github.com/blackecho/Deep-Learning-TensorFlow}}:
                Ready to use implementations of various Deep Learning algorithms using TensorFlow.
        \end{description}

        \section{Publications}
        \begin{description}
            \item G. Angeletti, T. Tommasi, B. Caputo.
                "Adaptive Deep Learning through Visual Domain Localization".
                In: \textit{IEEE International Conference on Robotics and Automation (ICRA)}(2018) \footnote{\url{https://arxiv.org/abs/1802.08833}}
        \end{description}

        \section{Languages}
        \begin{description}
            \item Italian, native speaker
            \item English, fluent spoken and written skills
        \end{description}

        \section{Honors}
        \begin{description}
            \item Ranked 13 / 2189 among Python github users in Italy by \footnote{\url{http://git-awards.com/}} \hfill 2018
            \item Winner of Accenture Digital Hackathon Rome \hfill 2016
            \item NASA International SpaceApps Challenge \hfill 2015\\
                Local winner (Rome) \& Global winner for category Galactic Impact\\
                Project: CROPP---Cultures Risks Observation and Prevention
                Platform\footnote{\url{https://2015.spaceappschallenge.org/award/}}
        \end{description}

        \section{Activities \& Hobbies}
        \begin{description}
        	\item Applications of machine learning to visual and musical arts
            \item Open source projects
            \item Data visualization
        \end{description}
    \end{resume}
\end{document}
